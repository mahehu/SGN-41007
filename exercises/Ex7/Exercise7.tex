\documentclass[a4paper,12pt]{scrartcl}

\usepackage{bm,amsmath,url,graphicx}
\usepackage{palatino}
\usepackage{color, xcolor}
\usepackage{listings}


\newcommand{\n}{\mathbf{ n}}
\newcommand{\h}{\mathbf{ h}}
\newcommand{\x}{\mathbf{ x}}
\newcommand{\y}{\mathbf{ y}}
\newcommand{\w}{\mathbf{ w}}
\newcommand{\HH}{\mathbf{ H}}
\newcommand{\R}{\mathbf{ R}}
\newcommand{\C}{\mathbf{ C}}
\newcommand{\thb}{{\boldsymbol{\theta}}}
\newcommand{\mub}{{\boldsymbol{\mu}}}
\newcommand{\python}{{\fbox{\texttt{\bfseries python}}\quad}}
\newcommand{\pen}{{\fbox{\texttt{\bfseries pen\&paper}}\quad}}

\renewcommand{\familydefault}{\rmdefault}


\begin{document}
\section*{SGN-41007 Pattern Recognition and Machine Learning}
\emph{Exercise Set 7: February 22--February 24, 2017}
\bigskip
\sloppy

\lstdefinestyle{mystyle}{
  belowcaptionskip=1\baselineskip,
  breaklines=true,
  frame=single,
  xleftmargin=\parindent,
  language=Python,
  showstringspaces=false,
  basicstyle=\ttfamily,
  keywordstyle=\bfseries\color{green!40!black},
  commentstyle=\itshape\color{purple!40!black},
  identifierstyle=\color{blue},
  stringstyle=\color{orange},
  moredelim=**[is][\color{red}]{@}{@},
}

\lstset{language=Python,style=mystyle} 


\noindent
Exercises consist of both pen\&paper and computer assignments.
Pen\&paper questions are solved at home before exercises, while
computer assignments are solved during exercise hours. The
computer assignments are marked by  \python and 
Pen\&paper questions by  \pen

\begin{enumerate}

\item \pen \emph{Error rate confidence limits.}

We train a classifier with a set of training examples, and test the 
accuracy of the resulting model with a set of $N=100$ test samples.
The classifier misclassifies $K=5$ of those. 

\begin{enumerate}
	\item Find the 90\% confidence interval of the result.
Hint: The classification accuracy can be modeled using
binomial distribution, whose confidence intervals are 
discussed here:

\url{https://en.wikipedia.org/wiki/Binomial_distribution#Confidence_intervals}

\item Another classifier misclassifies only 3 test samples.
Is it better than the first one with statistical significance
at 90\% confidence level?

\end{enumerate}

\item \pen  In Exercise set 5 (question 2a), we derived the formula for the gradient of log-loss.
\begin{enumerate}
	\item Compute the gradient for $L_2$ penalized log-loss.
	\item Study also the gradient for $L_1$ penalized log-loss. Propose an approximation, whose gradient would be defined for all $\w$.
\end{enumerate}

\item \python Implement the $L_2$ penalized log-loss minimizer in Python. You can use
the template of Question 3 at Exercise set 5.

\item \python Apply the recursive feature elimination approach 
(\verb+sklearn.feature_selection.RFECV+) with logistic regression classifier 
for the arcene dataset. The data can
be downloaded in \verb+*.mat+ format from:

{\small \url{http://www.cs.tut.fi/courses/SGN-41007/exercises/arcene.zip}}

Use \verb+scipy.io.loadmat+ to open the file. Note that your have to ravel
\verb+y_train+ and \verb+y_test+ so that \texttt{sklearn} will accept them.

\begin{enumerate}
	\item Instantiate an RFECV selector (call it \texttt{rfe} from now on). To speed up computation, set \texttt{step = 50} in the constructor.
	Also set \texttt{verbose = 1} to see the progress.
	\item Fit the RFECV to \verb+X_train+ and \verb+y_train+.
	\item Count the number of selected features from \verb+rfe.support_+.
	\item Plot the errors for different number of features:\\
	 \verb+plt.plot(range(0,10001,50), rfe.grid_scores_)+
	\item Compute the accuracy on \verb+X_test+ and \verb+y_test+. You can use \texttt{rfe} as any other classifier.
\end{enumerate}

\item \python Apply $L_1$ penalized Logistic Regression for feature selection with the arcene dataset.
Find a good value for parameter $C$ by 10-fold cross-validating the accuracy.
Study the sparseness of the solution: how many features were selected?

\begin{enumerate}
	\item Instantiate a LogisticRegression classifier. Set \texttt{penalty = 'l1'} in the constructor.
	\item Cross validate the accuracy of a range of \texttt{C} values (see earlier exercises).
	\item Fit the LogisticRegression to \verb+X_train+ and \verb+y_train+.
	\item Count the number of selected features from \verb+rfe.coef_+.
	\item Compute the accuracy on \verb+X_test+ and \verb+y_test+.
\end{enumerate}

\end{enumerate}

\end{document}          
